\chapter{Cơ sở lý thuyết và các công nghệ liên quan}
Chương này trình bày cơ sở lý thuyết và các công nghệ liên quan phục vụ việc xây dựng hệ thống IoT. Nội dung bao gồm tổng quan về IoT, các giao thức truyền thông phổ biến, các loại cảm biến và phần cứng sử dụng, cùng với các kỹ thuật bảo mật cần thiết trong quá trình truyền tải và xử lý dữ liệu.
\section{Tổng quan về Internet of Things}
Internet of Things (IoT) là thuật ngữ dùng để chỉ các đối tượng có thể được nhận biết cũng như sự tồn tại của chúng trong một kiến trúc mang tính kết nối. Đây là một viễn cảnh trong đó mọi vật, mọi con vật hoặc con người được cung cấp các định danh và khả năng tự động truyền tải dữ liệu qua một mạng lưới mà không cần sự tương tác giữa con người với con người hoặc con người với máy tính. IoT tiến hoá từ sự hội tụ của các công nghệ không dây, hệ thống vi cơ điện tử và Internet. 

"Thing" trong Internet of Things, có thể là một trang trại động vật với bộ tiếp sóng chip sinh học, một chiếc xe ô tô tích hợp các cảm biến để cảnh báo lái xe khi lốp quá non, hoặc bất kỳ đồ vật nào do tự nhiên sinh ra hoặc do con người sản xuất ra mà có thể được gán với một địa chỉ IP và được cung cấp khả năng truyền tải dữ liệu qua mạng lưới.

\section{Các giao thức truyền thông trong IoT}
\subsection{LoRa}
\subsubsection{Giới thiệu}
LoRa, viết tắt của Long Range Radio, là một loại công nghệ hỗ trợ truyền dữ liệu trong những khoảng cách lên đến hàng chục km mà không cần thêm bất kỳ các mạch khuếch đại công suất nào. LoRa giúp việc truyền và nhận dữ liệu trở nên đơn giản hơn, tiết kiệm năng lượng tiêu thụ hiệu quả.

Một mạng LoRa có thể cung cấp vùng phủ sóng tương tự như của một mạng di động. Trong một số trường hợp, các antenna Lora có thể kết hợp với antenna di động khi các tần số là gần nhau, từ đó giúp tiết kiệm đáng kể chi phí. 

\begin{figure}[H]
    \centering
    \includegraphics[width=0.55\linewidth]{images/chapter_2/LoRa.png}
    \caption{Minh hoạ về công nghệ LoRa}
    \label{fig:placeholder}
\end{figure}

\subsubsection{Nguyên lý hoạt động}
Nền tảng phát triển công nghệ LoRa dựa trên kỹ thuật điều chế Chirp Spread Spectrum. Khi các dữ liệu được tạo xung với tần số cao để tạo ra những tín hiệu có dải tần cao hơn. Các tín hiệu này sẽ được mã hóa theo các chuỗi chirp signal (tín hiệu hình sin thay đổi theo thời gian) trước khi được gửi đi từ antenna. Có hai loại chirp signal, bao gồm tần số up-chirp theo thời gian và tần số của down-chirp giảm dần theo thời gian.

Nguyên tắc hoạt động này hỗ trợ thiết bị giảm độ phức tạp và tăng độ chính xác cần thiết cho mạch nhận để có thể giải mã và điều chỉnh lại dữ liệu. LoRa không yêu cầu nhiều công suất phát mà vẫn có thể truyền đi xa, vì tín hiệu LoRa có thể nhận được ở khoảng cách xa ngay cả khi cường độ tín hiệu thấp hơn nhiễu xung quanh.

Băng tần hoạt động của LoRa nằm trong khoảng từ 430MHz đến 915MHz, áp dụng cho các khu vực khác nhau trên thế giới. Tín hiệu chirp sẽ cho phép các tín hiệu LoRa hoạt động trong cùng một khu vực mà không gây nhiễu lẫn nhau, cho phép nhiều thiết bị trao đổi dữ liệu trên nhiều kênh đồng thời.

\subsubsection{Đặc điểm nổi bật}
Là một công nghệ hiện đại được sử dụng phổ biến hiện nay, LoRa có khả năng truyền dữ liệu ở khoảng cách cực xa và có thể đạt khoảng cách truyền hơn 15km trong môi trường mở hoặc rộng hơn nữa. Nó còn có thể chạy với mức tiêu thụ điện năng thấp, điều này có thể kéo dài tuổi thọ pin và giảm chi phí sử dụng khi không cần thay quá nhiều lần.

Với kỹ thuật truyền của công nghệ LoRa, tốc độ truyền tuy thấp nhưng vẫn cung cấp đủ băng thông cho một số ứng dụng IoT nhất định, chẳng hạn như định vị, theo dõi tài nguyên và gửi thông tin trạng thái. Công nghệ này có khả năng chống nhiễu tốt và khả năng tự động tìm kiếm kênh truyền tốt nhất, giúp đảm bảo tính toàn vẹn của dữ liệu truyền đi.

\subsubsection{Ưu điểm và nhược điểm}
Nổi bật nhất, LoRa có mức tiêu thụ điện năng thấp, đây là ưu điểm lớn nhất của công nghệ LoRa. Bởi mức tiêu thụ điện năng của công nghệ này thấp. Tương ứng, tuổi thọ của ắc quy có thể lên đến 10 năm, hỗ trợ các nhà máy, doanh nghiệp giảm chi phí thay thế ắc quy.

LoRa còn có thể hỗ trợ máy tính truyền dữ liệu vài km mà không cần bộ khuếch đại công suất. Do Lora sử dụng ít nhiễu điện từ hơn nên tín hiệu có thể duy trì khoảng cách xa hoặc khả năng làm việc mạnh mẽ ngay cả trong môi trường đô thị với những ngôi nhà dày đặc. 

LoRa là một giao thức mạng mở, có khả năng cung cấp các kết nối nút cuối được tiêu chuẩn hóa giữa những máy tính và thiết bị IoT. Điều này cho phép mỗi nhà máy nhanh chóng triển khai các ứng dụng IoT ở mọi nơi. 

Công nghệ còn sở hữu mã hóa AES128, cho phép xác thực lẫn nhau, đảm bảo tính toàn vẹn và tăng tính bảo mật.

Tuy LoRa là một công nghệ được ưa chuộng sử dụng nhưng nó không phải một công nghệ có tính hoàn hảo về mọi mặt. Công nghệ này không phù hợp với những công việc cần tải dữ liệu lớn, đây cũng là nhược điểm lớn nhất đối với công nghệ LoRa. Do các sóng truyền ở tần số này làm chậm tốc độ truyền và tải trọng của công nghệ bị giới hạn ở 100 byte. Do đó, độ trễ của công nghệ LoRa sẽ cao hơn các phương pháp khác.

Khi sử dụng công nghệ LoRa, người dùng sẽ gặp khó khăn trong việc lắp đặt các gateway trong khu vực nội thành cũng là một trở ngại cho việc phổ cập công nghệ LoRa tại các khu vực đông dân cư.

LoRa có khả năng truyền dữ liệu hạn chế và không phù hợp với các ứng dụng yêu cầu truyền dữ liệu lớn. Ngoài ra, để có thể triển khai một hệ thống LoRa hoàn chỉnh, cần có nhiều cổng và thiết bị kết nối, điều này làm tăng chi phí triển khai.

\subsection{4G/5G}
\subsubsection{Giới thiệu}
Mạng 4G và 5G là các thế hệ mạng di động băng thông rộng được sử dụng rộng rãi trong các hệ thống IoT yêu cầu tốc độ truyền dữ liệu cao, độ trễ thấp và phạm vi phủ sóng lớn. 

Mạng 4G cung cấp tốc độ truyền tải cao hơn nhiều so với 3G, hỗ trợ tốt các ứng dụng IoT như camera giám sát, theo dõi phương tiện hoặc truyền dữ liệu thời gian thực. Mạng 5G là thế hệ mạng mới nhất với tốc độ vượt trội, độ trễ cực thấp, khả năng kết nối số lượng lớn thiết bị, đặc biệt phù hợp với các ứng dụng IoT quy mô lớn như xe tự hành, thành phố thông minh hoặc tự động hoá công nghiệp.
\subsubsection{Nguyên lý hoạt động}
Mạng 4G và 5G đều hoạt động dựa trên kiến trúc mạng di động tổ ong, trong đó mỗi khu vực được bao phủ bởi các trạm gốc truyền nhận tín hiệu vô tuyến. Khi một thiết bị IoT hoặc thiết bị di động khởi tạo kết nối, tín hiệu sẽ được truyền đến trạm gốc gần nhất để thực hiện quá trình xác thực và thiết lập phiên làm việc với mạng lõi. Đối với 4G, mạng lõi EPC chịu trách nhiệm định tuyến gói dữ liệu và đảm bảo chất lượng truyền dẫn. Trong khi đó, 5G sử dụng kiến trúc hiện đại hơn, cho phép xử lý dữ liệu linh hoạt và hiệu quả thông qua các chức năng ảo hóa và tách rời.

Sau khi kết nối được thiết lập, dữ liệu IoT được truyền dưới dạng gói IP qua trạm gốc, đi qua mạng lõi và đến server hoặc nền tảng cloud. Công nghệ 5G còn sử dụng các kỹ thuật tiên tiến như beamforming và Massive MIMO để tập trung tín hiệu theo hướng tối ưu, tăng tốc độ và giảm nhiễu. Ngoài ra, 5G hỗ trợ phân đoạn mạng, cho phép tạo ra nhiều “lát cắt” mạng độc lập nhằm phục vụ các nhóm ứng dụng có yêu cầu khác nhau, từ băng thông cao, độ trễ thấp cho đến số lượng kết nối lớn. Nhờ những cơ chế hoạt động này, 4G và đặc biệt là 5G có khả năng cung cấp tốc độ truyền tải nhanh, độ trễ thấp và hiệu suất kết nối ổn định cho các hệ thống IoT hiện đại.

\subsubsection{Đặc điểm nổi bật}
Công nghệ 4G và 5G đều mang lại những bước tiến quan trọng trong lĩnh vực truyền thông không dây, đặc biệt đối với các hệ thống IoT hiện đại. Ở thế hệ 4G, tốc độ truyền tải dữ liệu được nâng lên đáng kể, đạt đến 100 Mbps đối với thiết bị di động và có thể lên tới 1 Gbps trong các kết nối cố định. Độ trễ cũng được cải thiện, giảm xuống chỉ còn 5–20 ms, giúp phản hồi nhanh hơn cho các ứng dụng thời gian thực. 4G sử dụng các công nghệ như OFDMA và MIMO để tăng băng thông, hỗ trợ nhiều thiết bị kết nối đồng thời, đảm bảo hiệu suất ổn định trong môi trường đông người dùng.

Bước sang 5G, công nghệ này mang đến những đột phá vượt trội về tốc độ, độ trễ và khả năng kết nối. Tốc độ tải xuống lý thuyết có thể đạt tới 10 Gbps, nhanh hơn 4G từ 10 đến 100 lần, đồng thời độ trễ giảm xuống mức cực thấp chỉ khoảng 1 ms. Điều này cho phép 5G hỗ trợ tốt các ứng dụng yêu cầu phản hồi tức thời như xe tự hành, điều khiển robot từ xa, thực tế ảo. 5G cũng cho phép kết nối hàng triệu thiết bị trên mỗi km$^2$, đáp ứng nhu cầu của đô thị thông minh và các hệ thống IoT mật độ cao. Ngoài ra, nhờ ứng dụng sóng milimet, Massive MIMO và công nghệ phân mảnh mạng, 5G có khả năng tối ưu băng thông linh hoạt và tùy chỉnh tài nguyên theo từng loại dịch vụ, mang lại hiệu suất vượt trội trong nhiều tình huống sử dụng khác nhau.

\subsubsection{Ưu điểm và nhược điểm}
Công nghệ 4G và 5G mang lại nhiều lợi ích quan trọng trong các hệ thống IoT hiện đại. Với tốc độ truyền dữ liệu cao, 4G cho phép truyền tải video HD, chơi game trực tuyến và vận hành các dịch vụ yêu cầu băng thông rộng mà vẫn duy trì độ trễ thấp. Khi bước sang 5G, tốc độ còn được cải thiện vượt trội cùng với độ trễ cực thấp, giúp đáp ứng tốt các ứng dụng thời gian thực như điều khiển từ xa, xe tự hành hoặc các hệ thống giám sát thông minh. Chất lượng dịch vụ ổn định giúp mạng có thể xử lý đồng thời thoại, video và dữ liệu mà không ảnh hưởng đến hiệu năng. Ngoài ra, 5G còn hỗ trợ mật độ kết nối rất lớn, phù hợp cho các mô hình IoT quy mô cao trong đô thị thông minh hoặc công nghiệp.

Tuy nhiên, cả 4G và 5G đều tồn tại một số hạn chế. Chi phí triển khai hạ tầng cao khiến việc phủ sóng đồng đều gặp khó khăn, đặc biệt tại khu vực nông thôn hoặc vùng sâu vùng xa. Việc xây dựng mạng 5G càng yêu cầu đầu tư lớn hơn do sử dụng nhiều trạm thu phát và băng tần cao. Thiết bị di động sử dụng mạng 4G/5G thường tiêu tốn năng lượng lớn hơn, làm giảm thời gian sử dụng pin. Ngoài ra, dù đã có nhiều cải tiến về bảo mật, các mạng này vẫn đối mặt với rủi ro bị tấn công do kiến trúc phức tạp, đặc biệt ở 5G. Người dùng cũng cần có thiết bị tương thích mới để tận dụng đầy đủ lợi ích của 5G, gây khó khăn cho quá trình phổ cập.
\subsection{MQTT}
\subsubsection{Giới thiệu}
MQTT, viết tắt của Message Queuing Telemetry Transport, là giao thức truyền thông được sử dụng cho các thiết bị IoT với băng thông thấp, độ tin cậy cao và khả năng được sử dụng trong mạng lưới không ổn định. 

MQTT là một lựa chọn lý tưởng trong các môi trường mà giá mạng viễn thông đắt đỏ hoặc băng thông thấp hay thiếu tin cậy, hoặc khi chạy trên thiết bị nhúng bị giới hạn về tài nguyên tốc độ và bộ nhớ. Bởi vì giao thức này sử dụng băng thông thấp trong môi trường có độ trễ cao nên nó là một giao thức lý tưởng cho các ứng dụng M2M.

\subsubsection{Nguyên lý hoạt động}
MQTT hoạt động dựa trên mô hình Publish/Subscribe thông qua kiến trúc Client/Server, trong đó Broker đóng vai trò trung gian xử lý toàn bộ quá trình truyền thông. Các thiết bị IoT (Client) có thể gửi dữ liệu dưới dạng thông điệp đến một chủ đề với vai trò Publisher, hoặc đăng ký nhận thông điệp từ các chủ đề đó với vai trò Subscriber. Nhờ cơ chế tách biệt giữa nơi gửi và nơi nhận, MQTT giúp giảm tải băng thông, tối ưu tài nguyên và duy trì hiệu quả hoạt động ngay cả trong môi trường mạng yếu, rất phù hợp với các hệ thống IoT.

Broker được xem như lõi trung tâm của toàn bộ hệ thống, chịu trách nhiệm tiếp nhận thông điệp từ Publisher, xử lý và phân phối chúng đến các Subscriber tương ứng. Bên cạnh nhiệm vụ chính này, Broker còn có thể đảm nhận các chức năng bổ trợ như quản lý hàng đợi, lưu trữ thông điệp, ghi log hệ thống, xác thực và mã hóa nhằm đảm bảo an toàn trong truyền thông. Chính nhờ sự linh hoạt và cấu trúc nhẹ, MQTT trở thành một trong những giao thức phổ biến nhất trong các ứng dụng IoT hiện nay.

\begin{figure}[H]
    \centering
    \includegraphics[width=0.6\linewidth]{images/chapter_2/MQTT.jpg}
    \caption{Minh hoạ về nguyên lý hoạt động của MQTT}
    \label{fig:placeholder}
\end{figure}

\subsubsection{Đặc điểm nổi bật}
Công nghệ MQTT sở hữu nhiều đặc điểm nổi bật giúp nó trở thành một trong những giao thức tối ưu cho các hệ thống IoT hiện đại. Trước hết, MQTT sử dụng mô hình truyền thông Pub/Sub, cho phép việc trao đổi thông điệp diễn ra theo hướng phân tán và tách biệt hoàn toàn giữa bên gửi và bên nhận. Nhờ đó, việc truyền tải dữ liệu diễn ra ngay lập tức mà không phụ thuộc vào nội dung hay bản chất của thông điệp, giúp giảm độ phức tạp và tài nguyên xử lý trên thiết bị. Giao thức này hoạt động dựa trên nền tảng TCP/IP, đảm bảo độ tin cậy trong truyền dẫn và duy trì kết nối ổn định giữa các thiết bị IoT.

Một trong những ưu điểm quan trọng của MQTT là khả năng hỗ trợ ba mức chất lượng dịch vụ (Quality of Service, hay QoS), cho phép lựa chọn độ tin cậy phù hợp tùy theo ứng dụng. Ở mức QoS 0, thông điệp được gửi đúng một lần nhưng không có cơ chế xác nhận ngoài TCP/IP. Với QoS 1, phía gửi đảm bảo dữ liệu được nhận ít nhất một lần, có thể dẫn đến việc nhận trùng lặp thông điệp. Mức QoS 2 cung cấp độ tin cậy cao nhất khi đảm bảo mỗi thông điệp chỉ được nhận đúng một lần thông qua cơ chế bắt tay bốn bước. Bên cạnh đó, phần bao gói dữ liệu (packet) của MQTT rất nhỏ và được tối ưu đến mức tối thiểu, giúp giảm tải đáng kể cho đường truyền và phù hợp với các mạng băng thông thấp.

\subsubsection{Ưu điểm và nhược điểm}
MQTT mang lại nhiều ưu điểm quan trọng nhờ cơ chế hoạt động nhẹ, linh hoạt và phù hợp với đặc thù của các hệ thống IoT. Với mô hình Pub/Sub và phần bao gói dữ liệu nhỏ, MQTT giúp truyền tải thông tin hiệu quả, giảm đáng kể băng thông tiêu thụ và tối ưu việc sử dụng tài nguyên mạng. Tính chất hoạt động này cũng giúp tăng khả năng mở rộng hệ thống và duy trì hiệu suất tốt trong điều kiện mạng yếu hoặc hạn chế. MQTT rất phù hợp cho các ứng dụng điều khiển, giám sát từ xa như trong SCADA, IoT công nghiệp, hay các hệ thống thu thập dữ liệu cảm biến. 

Ngoài ra, giao thức có chi phí vận hành thấp, hỗ trợ bảo mật thông qua các lớp giao thức như TLS/SSL và được tin dùng bởi nhiều tập đoàn lớn. Việc phát triển ứng dụng dựa trên MQTT cũng được đơn giản hóa, giúp rút ngắn thời gian triển khai, đồng thời cơ chế Pub/Sub cho phép thu thập lượng dữ liệu lớn với băng thông ít hơn so với các giao thức truyền thống.

Tuy nhiên, MQTT cũng tồn tại một số hạn chế nhất định. Khi so sánh với CoAP, một giao thức hướng tài nguyên cho IoT, tốc độ truyền của MQTT đôi khi chậm hơn, đặc biệt trong các ứng dụng yêu cầu phản hồi tức thời. CoAP sử dụng mô hình tài nguyên tĩnh, trong khi MQTT hoạt động dựa trên cơ chế đăng ký động, khiến việc quản lý nội dung đôi khi phức tạp hơn. 

Một nhược điểm khác là MQTT không tích hợp sẵn cơ chế mã hóa dữ liệu, buộc hệ thống phải sử dụng các giao thức bảo mật bổ sung như TLS/SSL để đảm bảo an toàn. Cuối cùng, khả năng mở rộng ở quy mô rất lớn vẫn là thách thức, do Broker trở thành điểm trung tâm cần xử lý lượng lớn kết nối đồng thời, dễ dẫn đến quá tải nếu không được tối ưu đúng cách.

\subsection{So sánh và lựa chọn giao thức}
Việc lựa chọn giao thức truyền thông phù hợp đóng vai trò then chốt trong hệ thống IoT giám sát và điều phối vận chuyển đạn dược, nơi yêu cầu truyền dữ liệu ổn định, phạm vi xa, tiêu thụ năng lượng thấp và đảm bảo an toàn trong môi trường quân sự. Ba lựa chọn phổ biến trong các hệ thống IoT hiện nay là LoRa, 4G/5G và MQTT. Mỗi công nghệ mang những đặc trưng riêng và đáp ứng những nhóm yêu cầu khác nhau.

LoRa hoạt động ở dải tần ISM, sử dụng kỹ thuật trải phổ CSS, cho phép truyền dữ liệu tầm xa, tiêu thụ năng lượng rất thấp và chịu nhiễu tốt. Tốc độ truyền không cao nhưng phù hợp cho các gói dữ liệu cảm biến như GPS, nhiệt độ và cảnh báo rung/lắc. Điều này giúp LoRa đặc biệt phù hợp cho các thiết bị IoT chạy bằng pin và hoạt động trong môi trường di chuyển liên tục như phương tiện vận chuyển đạn dược.

Ngược lại, 4G/5G cung cấp tốc độ cao, độ trễ thấp và khả năng truyền dữ liệu thời gian thực. 4G thích hợp cho các hệ thống cần truyền dữ liệu liên tục như video, trong khi 5G hỗ trợ băng thông lớn, độ trễ chỉ vài mili-giây và khả năng kết nối thiết bị dày đặc. Tuy nhiên, cả 4G và 5G đều tiêu thụ năng lượng cao hơn đáng kể, phụ thuộc vào hạ tầng viễn thông và phát sinh chi phí vận hành (SIM, gói dữ liệu). Điều này làm chúng kém phù hợp hơn cho thiết bị IoT chạy pin và yêu cầu hoạt động dài ngày.

Trong khi đó, MQTT không phải là công nghệ truyền sóng, mà là giao thức tầng ứng dụng chạy trên TCP/IP, thường được dùng kết hợp với 4G/5G, Wi-Fi hoặc Ethernet. MQTT phù hợp cho truyền dữ liệu nhẹ, ổn định, nhưng không thể đảm bảo phạm vi rộng nếu lớp vật lý bên dưới không đáp ứng. Do đó, MQTT không phải lựa chọn thay thế cho LoRa hay 4G/5G mà chỉ đóng vai trò giao thức truyền tải khi thiết bị sử dụng mạng IP.

\begin{table}[H]
\centering
\begin{tabular}{|p{3cm}|p{3.3cm}|p{3.3cm}|p{3.3cm}|}
\hline
\textbf{Tiêu chí} & \textbf{LoRa} & \textbf{4G/5G} & \textbf{MQTT} \\ 
\hline

Phạm vi & 2--15 km, không phụ thuộc hạ tầng nhà mạng & Toàn quốc, phụ thuộc trạm BTS & Phụ thuộc mạng IP (4G/5G/Wi-Fi) \\ 
\hline

Năng lượng & Rất thấp & Cao & Trung bình (tuỳ mạng dưới) \\ 
\hline

Tốc độ & Thấp (0.3–50 kbps) & Cao đến rất cao (Mbps–Gbps) & Phụ thuộc 4G/5G/Wi-Fi \\ 
\hline

Loại dữ liệu & Cảm biến, GPS, cảnh báo & Video, dữ liệu lớn, real-time & Telemetry, log nhẹ \\ 
\hline

Độ ổn định & Cao ở vùng xa & Phụ thuộc hạ tầng & Phụ thuộc lớp mạng dưới \\ 
\hline

Chi phí & Rất thấp & Cao (SIM, data) & Thấp \\ 
\hline

\end{tabular}
\caption{So sánh giữa LoRa, 4G/5G và MQTT}
\end{table}

Dựa trên yêu cầu của hệ thống giám sát và điều phối vận chuyển đạn dược, bao gồm truyền dữ liệu cảm biến (GPS, nhiệt độ, rung/lắc), hoạt động tầm xa, độ ổn định cao, chi phí vận hành thấp và thiết bị tiêu thụ năng lượng nhỏ, nhóm quyết định lựa chọn LoRa làm giao thức truyền thông chính. LoRa đáp ứng tốt nhất các tiêu chí của môi trường quân sự, đặc biệt trong các tuyến đường xa khu vực dân cư, nơi hạ tầng viễn thông 4G/5G có thể không ổn định. 4G/5G chỉ được xem xét như tùy chọn bổ sung cho các khu vực có hạ tầng mạnh hoặc khi cần truyền dữ liệu dung lượng lớn.

\section{Các loại cảm biến, thiết bị sử dụng}
\subsection{ESP32 NodeMCU-32S CH340 Ai-Thinker}
ESP32 NodeMCU-32S CH340 Ai-Thinker được phát triển trên nền Vi điều khiển trung tâm là ESP32 SoC với công nghệ Wifi, BLE và kiến trúc ARM mới nhất hiện nay, kit có thiết kế phần cứng, firmware và cách sử dụng tương tự Kit NodeMCU ESP8266, với ưu điểm là cách sử dụng dễ dàng, ra chân đầy đủ, tích hợp mạch nạp và giao tiếp UART CH340, thích hợp với các nghiên cứu, ứng dụng về Wifi, BLE, IoT và điều khiển, thu thập dữ liệu qua mạng.

\begin{table}[H]
\centering
\begin{tabular}{|p{4cm}|p{8cm}|}
\hline
\textbf{Thông số} & \textbf{Giá trị} \\
\hline
SPI Flash & 32 Mbits \\
\hline
Dải tần số & 2400 - 2483.5 MHz \\
\hline
Bluetooth & BLE 4.2 BR/EDR \\
\hline
WiFi & 802.11 \\
\hline
Giao diện hỗ trợ & UART/SPI/SDIO/I2C/PWM/I2S/IR/\newline ADC/DAC \\
\hline
Nguồn sử dụng & 5VDC qua cổng Micro USB \\
\hline
Mạch nạp & Tích hợp CH340 UART \\
\hline
Số chân & 38 chân cắm 2.54mm, ra đầy đủ chân ESP32 \\
\hline
Tích hợp & LED trạng thái, nút nhấn IO0 (BOOT), nút ENABLE \\
\hline
Kích thước & 25.4 x 48.3 mm \\
\hline
\end{tabular}
\caption{Thông số kỹ thuật của ESP32 NodeMCU-32S CH340 Ai-Thinker}
\label{tab:esp32_specs}
\end{table}

\begin{figure}[H]
    \centering
    \includegraphics[width=0.4\linewidth]{images/chapter_2/ESP32.png}
    \caption{ESP32 NodeMCU-32S CH340 Ai-Thinker}
    \label{fig:placeholder}
\end{figure}

\subsection{Cảm biến nhiệt độ, độ ẩm DHT11}
DHT11 là cảm biến nhiệt độ, độ ẩm rất thông dụng hiện nay vì chi phí rẻ và rất dễ lấy dữ liệu thông qua giao tiếp 1-wire (giao tiếp digital 1-wire truyền dữ liệu duy nhất). Cảm biến được tích hợp bộ tiền xử lý tín hiệu giúp dữ liệu nhận về được chính xác mà không cần phải qua bất kỳ tính toán nào.

\begin{table}[H]
\centering
\begin{tabular}{|p{6cm}|p{4cm}|}
\hline
\textbf{Thông số} & \textbf{Giá trị} \\
\hline
Điện áp hoạt động & 3-5V\\
\hline
Phạm vi đo độ ẩm & 20 - 90\% RH\\
\hline 
Phạm vi đo nhiệt độ & $0$ - $50^\circ$C\\
\hline
Độ chính xác độ ẩm & $\pm 5\%$ RH \\
\hline
Độ chính xác nhiệt độ & $\pm 2^\circ$C \\
\hline
Tần số lấy mẫu tối đa & 1 Hz\\  
\hline
Khoảng cách truyền tối đa & 20m\\
\hline
\end{tabular}
\caption{Thông số kỹ thuật của cảm biến nhiệt độ, độ ẩm DHT11}
\label{tab:placeholder}
\end{table}

\begin{figure}[H]
    \centering
    \includegraphics[width=0.35\linewidth]{images/chapter_2/DHT11.jpg}
    \caption{Cảm biến nhiệt độ, độ ẩm DHT11}
    \label{fig:placeholder}
\end{figure}
\subsection{Cảm biến gia tốc 3 trục ADXL345}
Cảm biến gia tốc 3 trục ADXL345 được dùng để đo gia tốc hoặc độ rung theo ba trục trong hệ tọa độ Descartes. Thiết bị hỗ trợ giao tiếp I2C, dễ tích hợp và có nhiều thư viện mẫu đi kèm. Nhờ độ nhạy cao và kích thước nhỏ gọn, ADXL345 phù hợp với các ứng dụng di động. Cảm biến có thể đo được cả gia tốc tĩnh (dùng để xác định độ nghiêng dựa trên trọng lực) lẫn gia tốc động (phát hiện chuyển động, va đập hoặc rung động).

\begin{table}[H]
\centering
\begin{tabular}{|p{4cm}|p{6cm}|}
\hline
\textbf{Thông số} & \textbf{Giá trị} \\
\hline
Điện áp hoạt động & 3--5VDC \\
\hline
Điện áp giao tiếp & 3.3V \\
\hline
Dòng điện tiêu thụ & 30$\mu$A \\
\hline
Nhiệt độ hoạt động & $-40^oC$ - $85^oC$\\
\hline
Chuẩn giao tiếp & I2C, SPI \\
\hline
Dải đo gia tốc& $\pm2g$, $\pm4g$, $\pm8g$, $\pm16g$ \\
\hline
\end{tabular}
\caption{Thông số kỹ thuật của cảm biến gia tốc 3 trục ADXL345}
\label{tab:placeholder}
\end{table}


\begin{figure}[H]
    \centering
    \includegraphics[width=0.35\linewidth]{images/chapter_2/ADXL345.jpg}
    \caption{Cảm biến gia tốc 3 trục ADXL345}
    \label{fig:placeholder}
\end{figure}

\subsection{Module GPS NEO-6MV2}
Module GPS NEO-6MV2 là một module GPS hoàn chỉnh dựa trên GPS Ublox NEO 6M. Thiết bị này sử dụng công nghệ mới nhất của Ublox để cung cấp thông tin định vị tốt nhất có thể và bao gồm một ăng-ten GPS chủ động tích hợp lớn hơn với chân cắm UART TTL.

Module GPS Ublox có đầu ra TTL nối tiếp, đồng thời có đèn LED hiển thị trạng thái để dễ dàng quan sát trong quá trình sử dụng.

\begin{table}[H]
\centering
\begin{tabular}{|p{4.5cm}|p{3cm}|}
\hline
\textbf{Thông số} & \textbf{Giá trị} \\
\hline
Điện áp hoạt động & 3--5VDC \\
\hline
Điện áp giao tiếp & 3.3V \\
\hline
Kích thước antenna & 12 $\times$ 12mm \\
\hline
Kích thước module & 23 $\times$ 30mm \\
\hline
Chuẩn giao tiếp & UART TTL \\ 
\hline
Baud rate mặc định & 9600\\
\hline
\end{tabular}
\caption{Thông số kỹ thuật của module GPS NEO-6MV2}
\label{tab:placeholder}
\end{table}


\begin{figure}[H]
    \centering
    \includegraphics[width=0.35\linewidth]{images/chapter_2/GPS.jpg}
    \caption{Module GPS NEO-6MV2}
    \label{fig:placeholder}
\end{figure}

\subsection{Module truyền thông LoRa}
Kit phát triển LoraWan SoC ASR6601 868/915MHz Ra-08H Development Board Ai-Thinker được sản xuất bởi Ai-Thinker sử dụng chip SoC ASR6601 là sự kết hợp giữa MCU kiến trúc 32-bit RISC và chuẩn truyền sóng Lora, rất thích hợp cho các ứng dụng xây dựng mạng truyền sóng LoraWan, chuẩn Lora mang đến hai yếu tố quan trọng là tiết kiệm năng lượng và khoảng cách phát siêu xa, ngoài ra nó còn có khả năng cấu hình để tạo thành mạng truyền nhận nên hiện tại được phát triển và sử dụng rất nhiều trong các nghiên cứu về IoT.

\begin{figure}[H]
    \centering
    \includegraphics[width=0.35\linewidth]{images/chapter_2/Module LoRa.png}
    \caption{Module LoRa Ra-08H Development Board Ai-Thinker}
    \label{fig:placeholder}
\end{figure}
\section{Các kỹ thuật bảo mật trong IoT}
Bảo mật là yếu tố sống còn trong hệ thống giám sát vận chuyển hàng hóa quân sự. Đề tài tập trung vào ba khía cạnh bảo mật chính:

\subsection{Mã hóa dữ liệu (Confidentiality)}
Để đảm bảo dữ liệu (tọa độ, trạng thái đạn dược) không bị nghe lén trong quá trình truyền tải qua không gian, hệ thống sử dụng thuật toán mã hóa AES-128 (Advanced Encryption Standard). Trước khi gửi gói tin từ ESP32 qua LoRa hoặc 4G, dữ liệu payload sẽ được mã hóa bằng một khóa bí mật (Secret Key) chỉ được biết bởi thiết bị gửi và máy chủ nhận.

\subsection{Xác thực thiết bị (Authentication)}
Hệ thống ngăn chặn các thiết bị lạ giả mạo thành xe vận chuyển bằng cơ chế xác thực.
\begin{itemize}
    \item \textbf{Với MQTT:} Sử dụng Username/Password và Client ID định danh cho từng thiết bị khi kết nối vào Broker.
    \item \textbf{Với LoRa:} Sử dụng cơ chế xác thực ABP (Activation By Personalization) hoặc OTAA (Over-The-Air Activation) để đảm bảo chỉ các node đã đăng ký mới được tham gia mạng.
\end{itemize}

\subsection{Toàn vẹn dữ liệu (Integrity)}
Để đảm bảo dữ liệu không bị thay đổi hoặc làm sai lệch trên đường truyền, hệ thống sử dụng kỹ thuật Checksum hoặc HMAC (Hash-based Message Authentication Code). Phía nhận sẽ tính toán lại mã kiểm tra và so sánh với mã nhận được; nếu không khớp, gói tin sẽ bị loại bỏ để tránh xử lý sai thông tin.
