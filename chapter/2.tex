\chapter{Cơ sở lý thuyết và các công nghệ liên quan}
Chương này trình bày cơ sở lý thuyết và các công nghệ liên quan phục vụ việc xây dựng hệ thống IoT. Nội dung bao gồm tổng quan về IoT, các giao thức truyền thông phổ biến, các loại cảm biến và phần cứng sử dụng, cùng với các kỹ thuật bảo mật cần thiết trong quá trình truyền tải và xử lý dữ liệu.
\section{Tổng quan về Internet of Things}
Internet of Things (IoT) là thuật ngữ dùng để chỉ các đối tượng có thể được nhận biết cũng như sự tồn tại của chúng trong một kiến trúc mang tính kết nối. Đây là một viễn cảnh trong đó mọi vật, mọi con vật hoặc con người được cung cấp các định danh và khả năng tự động truyền tải dữ liệu qua một mạng lưới mà không cần sự tương tác giữa con người với con người hoặc con người với máy tính. IoT tiến hoá từ sự hội tụ của các công nghệ không dây, hệ thống vi cơ điện tử và Internet. 

"Thing" trong Internet of Things, có thể là một trang trại động vật với bộ tiếp sóng chip sinh học, một chiếc xe ô tô tích hợp các cảm biến để cảnh báo lái xe khi lốp quá non, hoặc bất kỳ đồ vật nào do tự nhiên sinh ra hoặc do con người sản xuất ra mà có thể được gán với một địa chỉ IP và được cung cấp khả năng truyền tải dữ liệu qua mạng lưới.

\section{Các giao thức truyền thông trong IoT}
\subsection{LoRa}
\subsubsection{Giới thiệu}
LoRa, viết tắt của Long Range Radio, là một loại công nghệ hỗ trợ truyền dữ liệu trong những khoảng cách lên đến hàng chục km mà không cần thêm bất kỳ các mạch khuếch đại công suất nào. LoRa giúp việc truyền và nhận dữ liệu trở nên đơn giản hơn, tiết kiệm năng lượng tiêu thụ hiệu quả.

Một mạng LoRa có thể cung cấp vùng phủ sóng tương tự như của một mạng di động. Trong một số trường hợp, các antenna Lora có thể kết hợp với antenna di động khi các tần số là gần nhau, do đó giúp tiết kiệm đáng kể chi phí. 

\begin{figure}[H]
    \centering
    \includegraphics[width=0.55\linewidth]{images/chapter_2/LoRa.png}
    \caption{Minh hoạ về công nghệ LoRa}
    \label{fig:placeholder}
\end{figure}

\subsubsection{Nguyên lý hoạt động}
Nền tảng phát triển công nghệ LoRa dựa trên kỹ thuật điều chế Chirp Spread Spectrum. Khi các dữ liệu được tạo xung với tần số cao để tạo ra những tín hiệu có dải tần cao hơn. Các tín hiệu này sẽ được mã hóa theo các chuỗi chirp signal (tín hiệu hình sin thay đổi theo thời gian) trước khi được gửi đi từ antenna. Có hai loại chirp signal, bao gồm tần số up-chirp theo thời gian và tần số của down-chirp giảm dần theo thời gian.

Nguyên tắc hoạt động này hỗ trợ thiết bị giảm độ phức tạp và tăng độ chính xác cần thiết cho mạch nhận để có thể giải mã và điều chỉnh lại dữ liệu. LoRa không yêu cầu nhiều công suất phát mà vẫn có thể truyền đi xa, vì tín hiệu LoRa có thể nhận được ở khoảng cách xa ngay cả khi cường độ tín hiệu thấp hơn nhiễu xung quanh.

Băng tần hoạt động của LoRa nằm trong khoảng từ 430MHz đến 915MHz, áp dụng cho các khu vực khác nhau trên thế giới. Tín hiệu chirp sẽ cho phép các tín hiệu LoRa hoạt động trong cùng một khu vực mà không gây nhiễu lẫn nhau, cho phép nhiều thiết bị trao đổi dữ liệu trên nhiều kênh đồng thời.

\subsubsection{Đặc điểm nổi bật}
Là một công nghệ hiện đại được sử dụng phổ biến hiện nay, LoRa có khả năng truyền dữ liệu ở khoảng cách cực xa và có thể đạt khoảng cách truyền hơn 15km trong môi trường mở hoặc rộng hơn nữa. Nó còn có thể chạy với mức tiêu thụ điện năng thấp, điều này có thể kéo dài tuổi thọ pin và giảm chi phí sử dụng khi không cần thay quá nhiều lần.

Với kỹ thuật truyền của công nghệ LoRa, tốc độ truyền tuy thấp nhưng vẫn cung cấp đủ băng thông cho một số ứng dụng IoT nhất định, chẳng hạn như định vị, theo dõi tài nguyên và gửi thông tin trạng thái. Công nghệ này có khả năng chống nhiễu tốt và khả năng tự động tìm kiếm kênh truyền tốt nhất, giúp đảm bảo tính toàn vẹn của dữ liệu truyền đi.

\subsubsection{Ưu điểm và nhược điểm}
Nổi bật nhất, LoRa có mức tiêu thụ điện năng thấp, đây là ưu điểm lớn nhất của công nghệ LoRa. Bởi mức tiêu thụ điện năng của công nghệ này thấp. Tương ứng, tuổi thọ của ắc quy có thể lên đến 10 năm, hỗ trợ các nhà máy, doanh nghiệp giảm chi phí thay thế ắc quy.

LoRa còn có thể hỗ trợ máy tính truyền dữ liệu vài km mà không cần bộ khuếch đại công suất. Do Lora sử dụng ít nhiễu điện từ hơn nên tín hiệu có thể duy trì khoảng cách xa hoặc khả năng làm việc mạnh mẽ ngay cả trong môi trường đô thị với những ngôi nhà dày đặc. 

LoRa là một giao thức mạng mở, có khả năng cung cấp các kết nối nút cuối được tiêu chuẩn hóa giữa những máy tính và thiết bị IoT. Điều này cho phép mỗi nhà máy nhanh chóng triển khai các ứng dụng IoT ở mọi nơi. 

Công nghệ còn sở hữu mã hóa AES128, cho phép xác thực lẫn nhau, đảm bảo tính toàn vẹn và tăng tính bảo mật.

Tuy LoRa là một công nghệ được ưa chuộng sử dụng nhưng nó không phải một công nghệ có tính hoàn hảo về mọi mặt. Công nghệ này không phù hợp với những công việc cần tải dữ liệu lớn, đây cũng là nhược điểm lớn nhất đối với công nghệ LoRa. Do các sóng truyền ở tần số này làm chậm tốc độ truyền và tải trọng của công nghệ bị giới hạn ở 100 byte. Do đó, độ trễ của công nghệ LoRa sẽ cao hơn các phương pháp khác.

Khi sử dụng công nghệ LoRa, người dùng sẽ gặp khó khăn trong việc lắp đặt các gateway trong khu vực nội thành cũng là một trở ngại cho việc phổ cập công nghệ LoRa tại các khu vực đông dân cư.

LoRa có khả năng truyền dữ liệu hạn chế và không phù hợp với các ứng dụng yêu cầu truyền dữ liệu lớn. Ngoài ra, để có thể triển khai một hệ thống LoRa hoàn chỉnh, cần có nhiều cổng và thiết bị kết nối, điều này làm tăng chi phí triển khai.

\subsection{4G/5G}

\subsubsection{Giới thiệu}
Công nghệ mạng di động thế hệ thứ 4 (4G - LTE) và thế hệ thứ 5 (5G) là các chuẩn truyền thông không dây băng thông rộng, cung cấp khả năng kết nối Internet tốc độ cao cho các thiết bị di động. Trong các hệ thống IoT giám sát vận tải, 4G/5G đóng vai trò là kênh truyền thông chủ đạo (backhaul) hoặc kênh dự phòng quan trọng, giúp duy trì kết nối liên tục giữa phương tiện đang di chuyển và máy chủ trung tâm, đặc biệt là tại những khu vực không có sóng LoRa hoặc khi cần truyền tải dữ liệu đa phương tiện (hình ảnh, video).

\subsubsection{Nguyên lý hoạt động}
Mạng 4G/5G hoạt động dựa trên cấu trúc các trạm thu phát sóng (BTS - Base Transceiver Station) được phân bố rộng khắp theo hình dạng tổ ong (cells).
\begin{itemize}
    \item \textbf{Mã hóa và điều chế:} Dữ liệu từ thiết bị IoT được mã hóa kỹ thuật số và điều chế (sử dụng kỹ thuật OFDM) để truyền qua sóng vô tuyến đến trạm BTS gần nhất.
    \item \textbf{Chuyển mạch gói (Packet Switching):} Khác với các mạng di động thế hệ cũ sử dụng chuyển mạch kênh, 4G/5G sử dụng chuyển mạch gói hoàn toàn trên nền tảng IP, giúp tối ưu hóa việc truyền dữ liệu.
    \item \textbf{Mạng lõi (Core Network):} Dữ liệu từ BTS được chuyển về mạng lõi để định tuyến, xác thực và chuyển tiếp đến đích là máy chủ ứng dụng hoặc Internet.
\end{itemize}

\subsubsection{Đặc điểm nổi bật}
\begin{itemize}
    \item \textbf{Tốc độ truyền tải cao:} 4G có thể đạt tốc độ lý thuyết lên tới hàng trăm Mbps, và 5G có thể đạt tới hàng chục Gbps, đáp ứng tốt nhu cầu truyền dữ liệu lớn thời gian thực.
    \item \textbf{Độ trễ thấp:} Đặc biệt với 5G, độ trễ có thể giảm xuống mức dưới 10ms, cực kỳ quan trọng đối với các ứng dụng cảnh báo tức thời như giám sát đạn dược.
    \item \textbf{Tính di động cao (Mobility):} Hỗ trợ duy trì kết nối ổn định ngay cả khi thiết bị di chuyển với tốc độ cao trên phương tiện vận tải.
    \item \textbf{Vùng phủ sóng rộng:} Hạ tầng mạng 4G hiện nay đã bao phủ hầu hết các khu vực địa lý, đảm bảo khả năng giám sát liên tục.
\end{itemize}

\subsubsection{Ưu điểm và nhược điểm}
\begin{itemize}
    \item \textbf{Ưu điểm:}
    \begin{itemize}
        \item Phạm vi phủ sóng toàn quốc, không cần tự xây dựng hạ tầng trạm thu phát như LoRaWAN.
        \item Băng thông lớn, cho phép gửi dữ liệu cập nhật firmware (OTA) hoặc gửi log chi tiết dễ dàng.
        \item Độ trễ thấp, đảm bảo tính thời gian thực (Real-time).
    \end{itemize}
    \item \textbf{Nhược điểm:}
    \begin{itemize}
        \item Tiêu thụ năng lượng cao hơn nhiều so với LoRa, đòi hỏi nguồn cấp ổn định từ phương tiện.
        \item Chi phí vận hành cao hơn do phải trả phí thuê bao dữ liệu cho nhà mạng.
        \item Phụ thuộc vào chất lượng sóng của nhà cung cấp dịch vụ viễn thông.
    \end{itemize}
\end{itemize}

% ============================================================
% 2.2.3 MQTT
% ============================================================
\subsection{MQTT}

\subsubsection{Giới thiệu}
MQTT (Message Queuing Telemetry Transport) là một giao thức truyền thông điệp dạng publish-subscribe (xuất bản - đăng ký) cực kỳ nhẹ, được thiết kế riêng cho các thiết bị có băng thông hạn chế và độ trễ mạng cao. Đây là giao thức chuẩn "de facto" trong lĩnh vực IoT, hoạt động trên nền tảng TCP/IP, giúp kết nối các cảm biến và thiết bị di động với máy chủ đám mây một cách hiệu quả.

\subsubsection{Nguyên lý hoạt động}
Mô hình hoạt động của MQTT bao gồm ba thành phần chính:
\begin{itemize}
    \item \textbf{Publisher (Người xuất bản):} Là thiết bị IoT (trong dự án này là bộ thiết bị trên xe chở đạn) thu thập dữ liệu cảm biến và gửi đi.
    \item \textbf{Broker (Môi giới):} Là máy chủ trung gian (ví dụ: Mosquitto, EMQX). Nhiệm vụ của Broker là nhận thông điệp từ Publisher và lọc, phân phối lại cho các Subscriber quan tâm.
    \item \textbf{Subscriber (Người đăng ký):} Là ứng dụng giám sát hoặc backend server đăng ký nhận tin từ Broker.
\end{itemize}

Cơ chế hoạt động dựa trên \textbf{Topic (Chủ đề):} Khi thiết bị gửi dữ liệu, nó sẽ gửi vào một Topic cụ thể (ví dụ: \texttt{vehicle/truck01/temp}). Ứng dụng giám sát sẽ đăng ký Topic này để nhận dữ liệu ngay lập tức.

% Bạn có thể bỏ comment dòng dưới để chèn hình ảnh nếu có file ảnh
% \begin{figure}[h]
%     \centering
%     \includegraphics[width=0.8\textwidth]{images/mqtt_architecture.png}
%     \caption{Kiến trúc mô hình Publish-Subscribe của MQTT}
%     \label{fig:mqtt_arch}
% \end{figure}

\subsubsection{Đặc điểm nổi bật}
\begin{itemize}
    \item \textbf{Gọn nhẹ (Lightweight):} Header của gói tin MQTT rất nhỏ (tối thiểu chỉ 2 byte), giúp tiết kiệm băng thông 3G/4G và năng lượng.
    \item \textbf{Chất lượng dịch vụ (QoS - Quality of Service):} MQTT cung cấp 3 mức độ tin cậy khi truyền tin, rất quan trọng cho giám sát hàng nguy hiểm:
    \begin{itemize}
        \item \textit{QoS 0:} Gửi tối đa 1 lần (có thể mất tin).
        \item \textit{QoS 1:} Gửi ít nhất 1 lần (đảm bảo tin đến nơi, nhưng có thể lặp).
        \item \textit{QoS 2:} Gửi chính xác 1 lần (đảm bảo tin đến và không lặp).
    \end{itemize}
    \item \textbf{Last Will and Testament (LWT):} Tính năng giúp Broker thông báo cho Server biết nếu thiết bị đột ngột mất kết nối (ví dụ: xe đi vào vùng mất sóng hoặc bị ngắt nguồn).
    \item \textbf{Retained Messages:} Giúp thiết bị mới kết nối nhận được ngay trạng thái gần nhất của hệ thống.
\end{itemize}

\subsubsection{Ưu điểm và nhược điểm}
\begin{itemize}
    \item \textbf{Ưu điểm:}
    \begin{itemize}
        \item Sử dụng băng thông mạng cực kỳ hiệu quả, giảm chi phí 4G.
        \item Hoạt động ổn định ngay cả trong điều kiện mạng chập chờn, độ trễ cao.
        \item Dễ dàng mở rộng hệ thống (Scalability) lên hàng ngàn thiết bị mà không cần thay đổi kiến trúc.
        \item Cơ chế bảo mật hỗ trợ xác thực (username/password) và mã hóa SSL/TLS.
    \end{itemize}
    \item \textbf{Nhược điểm:}
    \begin{itemize}
        \item Chạy trên nền TCP nên vẫn có độ trễ thiết lập kết nối cao hơn so với UDP (như CoAP).
        \item Mô hình tập trung vào Broker, nếu Broker gặp sự cố thì toàn bộ hệ thống truyền tin sẽ bị gián đoạn (trừ khi thiết lập Cluster).
    \end{itemize}
\end{itemize}

% ============================================================
% 2.2.4 So sánh và lựa chọn giao thức
% ============================================================
\subsection{So sánh và lựa chọn giao thức}

\subsubsection{So sánh các công nghệ truyền thông}
Dựa trên các phân tích kỹ thuật ở trên, Bảng \ref{tab:so_sanh_protocol} dưới đây tóm tắt sự khác biệt giữa các công nghệ truyền thông phổ biến trong IoT là LoRa, 4G/5G và Wi-Fi để làm cơ sở cho việc lựa chọn.

\begin{table}[h]
    \centering
    \caption{Bảng so sánh các công nghệ truyền thông IoT}
    \label{tab:so_sanh_protocol}
    \begin{tabular}{|p{3cm}|p{3.5cm}|p{3.5cm}|p{3.5cm}|}
    \hline
    \textbf{Tiêu chí} & \textbf{LoRa} & \textbf{4G/5G} & \textbf{Wi-Fi} \\ \hline
    \textbf{Phạm vi phủ sóng} & Rất xa (lên đến 15km+) [119] & Rộng (theo hạ tầng nhà mạng) & Ngắn (vài chục mét) \\ \hline
    \textbf{Tiêu thụ năng lượng} & Rất thấp (pin 5-10 năm) [126] & Cao & Trung bình/Cao \\ \hline
    \textbf{Băng thông} & Thấp (giới hạn $\approx$ 100 bytes) [134] & Rất cao (Mbps - Gbps) & Rất cao \\ \hline
    \textbf{Chi phí vận hành} & Thấp (không phí thuê bao) & Cao (cước phí data 3G/4G) & Thấp (tại điểm truy cập) \\ \hline
    \textbf{Độ trễ} & Cao & Thấp & Rất thấp \\ \hline
    \textbf{Khả năng triển khai} & Mạng riêng linh hoạt (Private Network) & Phụ thuộc hạ tầng viễn thông & Phụ thuộc điểm truy cập cục bộ \\ \hline
    \end{tabular}
\end{table}

\subsubsection{Lựa chọn và giải thích}
Dựa trên yêu cầu đặc thù của đề tài "Ứng dụng IoT trong giám sát và điều phối vận chuyển đạn dược", nhóm thực hiện quyết định lựa chọn công nghệ \textbf{LoRa} làm giao thức truyền thông chính cho các thiết bị giám sát gắn trên phương tiện, kết hợp với giao thức MQTT ở tầng ứng dụng. Các lý do chính cho sự lựa chọn này bao gồm:

\begin{itemize}
    \item \textbf{Phù hợp với đặc điểm dữ liệu:} Hệ thống giám sát vận chuyển đạn dược chủ yếu thu thập các thông số dạng chuỗi số liệu nhỏ như: toạ độ GPS, nhiệt độ, độ ẩm và thông số rung động [56]. Các gói tin này có kích thước nhỏ, hoàn toàn phù hợp với giới hạn tải trọng (payload) của LoRa [121, 134].
    
    \item \textbf{Tối ưu hoá năng lượng cho thiết bị di động:} Quá trình vận chuyển có thể kéo dài nhiều giờ hoặc nhiều ngày. LoRa có ưu điểm vượt trội về tiết kiệm năng lượng [124], cho phép thiết bị giám sát hoạt động bền bỉ bằng pin mà không cần nguồn cấp điện liên tục từ xe, đảm bảo an toàn cháy nổ trong môi trường vận chuyển đạn dược.
    
    \item \textbf{Khả năng hoạt động độc lập và bảo mật:} Với LoRa, nhóm có thể thiết lập một mạng riêng (Private LoRaWAN) mà không phụ thuộc hoàn toàn vào sóng di động của nhà mạng [129]. Điều này đặc biệt quan trọng trong các khu vực quân sự hoặc vùng sâu vùng xa nơi sóng 4G có thể không ổn định. Ngoài ra, chuẩn mã hoá AES128 tích hợp sẵn của LoRa giúp tăng cường tính bảo mật cho dữ liệu nhạy cảm [131].
    
    \item \textbf{Chi phí triển khai thấp:} Các module LoRa (như Ra-02) và vi điều khiển ESP32 có chi phí thấp, dễ dàng thay thế và bảo trì, phù hợp với phạm vi triển khai thử nghiệm của đề tài [154].
\end{itemize}

Mặc dù LoRa có nhược điểm về độ trễ và băng thông thấp so với 4G/5G, nhưng với mục tiêu giám sát trạng thái môi trường và vị trí (không yêu cầu truyền hình ảnh/video trực tiếp), sự đánh đổi này là hợp lý để đạt được sự ổn định về năng lượng và khoảng cách truyền tin.
\subsection{ESP32 NodeMCU-32S CH340 Ai-Thinker}
ESP32 NodeMCU-32S CH340 Ai-Thinker được phát triển trên nền Vi điều khiển trung tâm là ESP32 SoC với công nghệ Wifi, BLE và kiến trúc ARM mới nhất hiện nay, kit có thiết kế phần cứng, firmware và cách sử dụng tương tự Kit NodeMCU ESP8266, với ưu điểm là cách sử dụng dễ dàng, ra chân đầy đủ, tích hợp mạch nạp và giao tiếp UART CH340, thích hợp với các nghiên cứu, ứng dụng về Wifi, BLE, IoT và điều khiển, thu thập dữ liệu qua mạng.

\begin{table}[H]
\centering
\begin{tabular}{|p{4cm}|p{8cm}|}
\rowcolor{gray!15}
\hline
\textbf{Thông số} & \textbf{Giá trị} \\
\hline
SPI Flash & 32 Mbits \\
\hline
Dải tần số & 2400 - 2483.5 MHz \\
\hline
Bluetooth & BLE 4.2 BR/EDR \\
\hline
WiFi & 802.11 \\
\hline
Giao diện hỗ trợ & UART/SPI/SDIO/I2C/PWM/I2S/IR/\newline ADC/DAC \\
\hline
Nguồn sử dụng & 5VDC qua cổng Micro USB \\
\hline
Mạch nạp & Tích hợp CH340 UART \\
\hline
Số chân & 38 chân cắm 2.54mm, ra đầy đủ chân ESP32 \\
\hline
Tích hợp & LED trạng thái, nút nhấn IO0 (BOOT), nút ENABLE \\
\hline
Kích thước & 25.4 x 48.3 mm \\
\hline
\end{tabular}
\caption{Thông số kỹ thuật của ESP32 NodeMCU-32S CH340 Ai-Thinker}
\label{tab:esp32_specs}
\end{table}

\begin{figure}[H]
    \centering
    \includegraphics[width=0.4\linewidth]{images/chapter_2/ESP32.png}
    \caption{ESP32 NodeMCU-32S CH340 Ai-Thinker}
    \label{fig:placeholder}
\end{figure}

\subsection{Cảm biến nhiệt độ, độ ẩm DHT11}
DHT11 là cảm biến nhiệt độ, độ ẩm rất thông dụng hiện nay vì chi phí rẻ và rất dễ lấy dữ liệu thông qua giao tiếp 1-wire (giao tiếp digital 1-wire truyền dữ liệu duy nhất). Cảm biến được tích hợp bộ tiền xử lý tín hiệu giúp dữ liệu nhận về được chính xác mà không cần phải qua bất kỳ tính toán nào.

\begin{table}[H]
\centering
\begin{tabular}{|p{6cm}|p{4cm}|}
\rowcolor{gray!15}
\hline
\textbf{Thông số} & \textbf{Giá trị} \\
\hline
Điện áp hoạt động & 3-5V\\
\hline
Phạm vi đo độ ẩm & 20 - 90\% RH\\
\hline 
Phạm vi đo nhiệt độ & $0$ - $50^\circ$C\\
\hline
Độ chính xác độ ẩm & $\pm 5\%$ RH \\
\hline
Độ chính xác nhiệt độ & $\pm 2^\circ$C \\
\hline
Tần số lấy mẫu tối đa & 1 Hz\\  
\hline
Khoảng cách truyền tối đa & 20m\\
\hline
\end{tabular}
\caption{Thông số kỹ thuật của cảm biến nhiệt độ, độ ẩm DHT11}
\label{tab:placeholder}
\end{table}

\begin{figure}[H]
    \centering
    \includegraphics[width=0.35\linewidth]{images/chapter_2/DHT11.jpg}
    \caption{Cảm biến nhiệt độ, độ ẩm DHT11}
    \label{fig:placeholder}
\end{figure}
\subsection{Cảm biến gia tốc 3 trục ADXL345}
Cảm biến gia tốc 3 trục ADXL345 được dùng để đo gia tốc hoặc độ rung theo ba trục trong hệ tọa độ Descartes. Thiết bị hỗ trợ giao tiếp I2C, dễ tích hợp và có nhiều thư viện mẫu đi kèm. Nhờ độ nhạy cao và kích thước nhỏ gọn, ADXL345 phù hợp với các ứng dụng di động. Cảm biến có thể đo được cả gia tốc tĩnh (dùng để xác định độ nghiêng dựa trên trọng lực) lẫn gia tốc động (phát hiện chuyển động, va đập hoặc rung động).

\begin{table}[H]
\centering
\begin{tabular}{|p{4cm}|p{6cm}|}
\rowcolor{gray!15}
\hline
\textbf{Thông số} & \textbf{Giá trị} \\
\hline
Điện áp hoạt động & 3--5VDC \\
\hline
Điện áp giao tiếp & 3.3V \\
\hline
Dòng điện tiêu thụ & 30$\mu$A \\
\hline
Nhiệt độ hoạt động & $-40^oC$ - $85^oC$\\
\hline
Chuẩn giao tiếp & I2C, SPI \\
\hline
Dải đo gia tốc& $\pm2g$, $\pm4g$, $\pm8g$, $\pm16g$ \\
\hline
\end{tabular}
\caption{Thông số kỹ thuật của cảm biến gia tốc 3 trục ADXL345}
\label{tab:placeholder}
\end{table}


\begin{figure}[H]
    \centering
    \includegraphics[width=0.35\linewidth]{images/chapter_2/ADXL345.jpg}
    \caption{Cảm biến gia tốc 3 trục ADXL345}
    \label{fig:placeholder}
\end{figure}

\subsection{Module GPS NEO-6MV2}
Module GPS NEO-6MV2 là một module GPS hoàn chỉnh dựa trên GPS Ublox NEO 6M. Thiết bị này sử dụng công nghệ mới nhất của Ublox để cung cấp thông tin định vị tốt nhất có thể và bao gồm một ăng-ten GPS chủ động tích hợp lớn hơn với chân cắm UART TTL.

Module GPS Ublox có đầu ra TTL nối tiếp, đồng thời có đèn LED hiển thị trạng thái để dễ dàng quan sát trong quá trình sử dụng.

\begin{table}[H]
\centering
\begin{tabular}{|p{4.5cm}|p{3cm}|}
\rowcolor{gray!15}
\hline
\textbf{Thông số} & \textbf{Giá trị} \\
\hline
Điện áp hoạt động & 3--5VDC \\
\hline
Điện áp giao tiếp & 3.3V \\
\hline
Kích thước antenna & 12 $\times$ 12mm \\
\hline
Kích thước module & 23 $\times$ 30mm \\
\hline
Chuẩn giao tiếp & UART TTL \\ 
\hline
Baud rate mặc định & 9600\\
\hline
\end{tabular}
\caption{Thông số kỹ thuật của module GPS NEO-6MV2}
\label{tab:placeholder}
\end{table}


\begin{figure}[H]
    \centering
    \includegraphics[width=0.35\linewidth]{images/chapter_2/GPS.jpg}
    \caption{Module GPS NEO-6MV2}
    \label{fig:placeholder}
\end{figure}

\subsection{Module truyền thông LoRa Ra-08H Kit (ASR6601)}
Để đảm bảo khả năng truyền dữ liệu tầm xa với độ ổn định cao, đề tài sử dụng Kit phát triển Ra-08H do Ai-Thinker sản xuất. Module này được xây dựng dựa trên SoC (System on Chip) ASR6601, tích hợp vi xử lý 32-bit ARM Cortex-M4 và chip thu phát LoRa trong cùng một gói phần cứng.

Khác với các module thế hệ cũ chỉ đóng vai trò thu phát vật lý (như SX1278), Ra-08H là một module thông minh có khả năng tự xử lý giao thức LoRaWAN. Trong hệ thống này, Ra-08H kết nối với vi điều khiển trung tâm ESP32 thông qua giao tiếp UART, hoạt động ở dải tần số cao (868/915 MHz) giúp tín hiệu xuyên vật cản tốt hơn và phù hợp với các chuẩn công nghiệp hiện đại. Đặc biệt, phiên bản "H" (High Power) hỗ trợ công suất phát lên tới +22dBm, giúp tối ưu hóa khoảng cách truyền trong điều kiện vận chuyển thực tế.

\begin{table}[h]
    \centering
    \caption{Thông số kỹ thuật của module LoRa Ra-08H Kit}
    \label{tab:spec_ra08h}
    \begin{tabular}{|p{5cm}|p{5cm}|}
    \hline
    \textbf{Thông số} & \textbf{Giá trị} \\ \hline
    SoC điều khiển & ASR6601CB (ARM Cortex-M4) \\ \hline
    Dải tần số hỗ trợ & 803 MHz $\sim$ 930 MHz (868/915 MHz) \\ \hline
    Công suất phát (Tx) & Tối đa +22 dBm \\ \hline
    Độ nhạy thu (Rx) & -140 dBm \\ \hline
    Giao tiếp với ESP32 & UART (AT Commands) \\ \hline
    Điện áp hoạt động & 3.3V (cấp qua cổng USB hoặc chân pin) \\ \hline
    Khoảng cách truyền & > 10km (điều kiện lý tưởng) \\ \hline
    Bộ nhớ & 128KB Flash, 16KB SRAM \\ \hline
    \end{tabular}
\end{table}

\section{Các kỹ thuật bảo mật trong IoT}
Bảo mật là yếu tố sống còn trong hệ thống giám sát vận chuyển hàng hóa quân sự. Đề tài tập trung vào ba khía cạnh bảo mật chính:

\subsection{Mã hóa dữ liệu (Confidentiality)}
Để đảm bảo dữ liệu (tọa độ, trạng thái đạn dược) không bị nghe lén trong quá trình truyền tải qua không gian, hệ thống sử dụng thuật toán mã hóa AES-128 (Advanced Encryption Standard). Trước khi gửi gói tin từ ESP32 qua LoRa hoặc 4G, dữ liệu payload sẽ được mã hóa bằng một khóa bí mật (Secret Key) chỉ được biết bởi thiết bị gửi và máy chủ nhận.

\subsection{Xác thực thiết bị (Authentication)}
Hệ thống ngăn chặn các thiết bị lạ giả mạo thành xe vận chuyển bằng cơ chế xác thực.
\begin{itemize}
    \item \textbf{Với MQTT:} Sử dụng Username/Password và Client ID định danh cho từng thiết bị khi kết nối vào Broker.
    \item \textbf{Với LoRa:} Sử dụng cơ chế xác thực ABP (Activation By Personalization) hoặc OTAA (Over-The-Air Activation) để đảm bảo chỉ các node đã đăng ký mới được tham gia mạng.
\end{itemize}

\subsection{Toàn vẹn dữ liệu (Integrity)}
Để đảm bảo dữ liệu không bị thay đổi hoặc làm sai lệch trên đường truyền, hệ thống sử dụng kỹ thuật Checksum hoặc HMAC (Hash-based Message Authentication Code). Phía nhận sẽ tính toán lại mã kiểm tra và so sánh với mã nhận được; nếu không khớp, gói tin sẽ bị loại bỏ để tránh xử lý sai thông tin.
