\chapter{Giới thiệu}
Chương này trình bày tổng quan về đề tài, bao gồm bối cảnh, các vấn đề thực tiễn đặt ra và lý do cần thiết phải xây dựng hệ thống giám sát vận chuyển đạn dược ứng dụng IoT. Bên cạnh đó, chương cũng nêu rõ mục tiêu nghiên cứu, phạm vi triển khai và cấu trúc tổng thể của báo cáo. Những nội dung này đóng vai trò định hướng cho toàn bộ đề tài, giúp người đọc hiểu được mục đích, giới hạn và cách thức tổ chức của các phần tiếp theo.
\section{Bối cảnh của đề tài}
Trong bối cảnh hiện nay, nhu cầu hiện đại hóa các hoạt động quản lý và đảm bảo an toàn trong lĩnh vực quân sự ngày càng trở nên cấp thiết. Trong đó, công tác vận chuyển đạn dược là một quy trình có độ rủi ro cao, đòi hỏi sự giám sát chặt chẽ về vị trí, tình trạng môi trường và an toàn kỹ thuật trong suốt quá trình di chuyển. Tuy nhiên, phần lớn hoạt động theo dõi hiện nay vẫn phụ thuộc vào ghi chép thủ công hoặc các phương pháp quản lý truyền thống, dẫn đến hạn chế trong khả năng giám sát theo thời gian thực, chậm trễ trong xử lý sự cố và thiếu tính minh bạch trong quản lý hành trình.

Sự phát triển của IoT đã mở ra khả năng tự động hóa giám sát, thu thập dữ liệu liên tục và truyền tải thông tin theo thời gian thực. IoT cho phép tích hợp các cảm biến đo rung, nhiệt độ, vị trí và các tín hiệu khác vào một hệ thống tập trung, hỗ trợ đơn vị quản lý phát hiện bất thường và đưa ra cảnh báo kịp thời. Điều này góp phần nâng cao tính an toàn, độ tin cậy và hiệu quả trong quy trình vận chuyển đạn dược, vốn là một hoạt động đặc biệt nhạy cảm và quan trọng.

\section{Mục tiêu nghiên cứu}

\subsection{Mục tiêu tổng quát}
Nghiên cứu và xây dựng hệ thống giám sát vận tải quân sự dựa trên công nghệ LoRa P2P (Point-to-Point) và vi điều khiển ESP32, đảm bảo khả năng truyền tin tin cậy trong điều kiện thiếu hạ tầng mạng.

\subsection{Mục tiêu cụ thể}
Để hiện thực hóa mục tiêu tổng quát nêu trên, đề tài tập trung giải quyết ba nhiệm vụ cốt lõi sau:

Thứ nhất, \textbf{nghiên cứu và tối ưu hóa kiến trúc phần mềm nhúng} cho giao thức truyền tin LoRa. Trọng tâm là đảm bảo tính thời gian thực (real-time) và tối ưu hóa năng lượng tiêu thụ cho thiết bị đầu cuối hoạt động bằng pin.

Thứ hai, \textbf{thiết kế khối thu thập dữ liệu} tích hợp đa cảm biến. Hệ thống sẽ sử dụng cảm biến DHT11 để giám sát nhiệt độ, độ ẩm; module Neo-6M cho định vị toạ độ GPS; và gia tốc kế ADXL345 nhằm phát hiện các va đập bất thường có thể ảnh hưởng đến an toàn của khí tài.

Thứ ba, \textbf{xây dựng cơ sở hạ tầng Back-end và giao diện Frontend}. Hệ thống cần có khả năng thu thập dữ liệu tập trung thông qua giao thức MQTT và hiển thị trực quan các thông số giám sát trên Dashboard (sử dụng nền tảng Firebase và Web App) để phục vụ công tác chỉ huy, điều hành.

\section{Đối tượng và phạm vi nghiên cứu}

\subsection{Đối tượng nghiên cứu}
Đề tài tập trung nghiên cứu sâu vào hai nhóm đối tượng chính:
\begin{itemize}
    \item \textbf{Về phần cứng:} Vi điều khiển ESP32 đóng vai trò bộ xử lý trung tâm, Module LoRa RA-08H cho truyền thông tầm xa, và các module cảm biến ngoại vi (GPS, DHT11, ADXL345).
    \item \textbf{Về giải pháp phần mềm:} Các giao thức truyền thông điệp tin cậy (LoRa P2P, MQTT), kỹ thuật lập trình nhúng trên Arduino IDE, ngôn ngữ Python cho các tác vụ cầu nối (bridge), và cơ sở dữ liệu thời gian thực Firebase.
\end{itemize}

\subsection{Phạm vi triển khai của đề tài}
Trong khuôn khổ của đồ án tốt nghiệp, phạm vi nghiên cứu được giới hạn ở các nội dung sau:
\begin{enumerate}
    \item \textbf{Môi trường phát triển:} Tập trung phát triển và kiểm thử trên bộ Kit phần cứng mô phỏng, chưa triển khai trên phương tiện vận tải thực tế quy mô lớn.
    \item \textbf{Kịch bản thử nghiệm:} Thực hiện đo đạc khả năng truyền tin trong phạm vi hẹp (phòng thí nghiệm) và mô phỏng các kịch bản mất kết nối để đánh giá tính năng lưu trữ cục bộ (Local Logging) trên thẻ nhớ SD.
    \item \textbf{Giới hạn nghiên cứu:} Đề tài tập trung giải quyết bài toán ở tầng ứng dụng (Application Layer) và tầng truyền thông (Network Layer), chưa đi sâu vào các giải pháp bảo mật vật lý chuyên dụng hay thiết kế cơ khí chịu lực cho vỏ hộp thiết bị.
\end{enumerate}

\section{Cấu trúc của đề tài}
Đề tài được tổ chức thành bảy chương theo trình tự logic giúp người đọc dễ dàng theo dõi quá trình nghiên cứu và triển khai hệ thống. Mỗi chương đảm nhận một vai trò riêng, từ việc giới thiệu bối cảnh cho đến trình bày lý thuyết nền tảng, phân tích yêu cầu, thiết kế, cài đặt, đánh giá và tổng kết. Cấu trúc này đảm bảo tính mạch lạc và phản ánh đầy đủ các bước thực hiện của đồ án.

\textbf{Chương 1 - Giới thiệu:} Tổng quan về đề tài, lý do chọn đề tài và mục tiêu nghiên cứu.

\textbf{Chương 2 - Cơ sở lý thuyết:} Trình bày về công nghệ LoRa, vi điều khiển ESP32 và các giao thức liên quan.

\textbf{Chương 3 - Phân tích yêu cầu:} Xác định các yêu cầu phi chức năng và chức năng của hệ thống.

\textbf{Chương 4 - Thiết kế hệ thống:} Thiết kế kiến trúc phần cứng và phần mềm, sơ đồ khối và luồng dữ liệu.

\textbf{Chương 5 - Cài đặt và Triển khai:} Mô tả quá trình lập trình, lắp ráp mạch và cấu hình hệ thống.

\textbf{Chương 6 - Kết quả và Đánh giá:} Trình bày kết quả thử nghiệm thực tế và đánh giá hiệu năng.

\textbf{Chương 7 - Kết luận và Hướng phát triển:} Tổng kết các kết quả đạt được và đề xuất hướng nghiên cứu tiếp theo.